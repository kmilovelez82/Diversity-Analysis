% Options for packages loaded elsewhere
\PassOptionsToPackage{unicode}{hyperref}
\PassOptionsToPackage{hyphens}{url}
%
\documentclass[
]{article}
\usepackage{lmodern}
\usepackage{amssymb,amsmath}
\usepackage{ifxetex,ifluatex}
\ifnum 0\ifxetex 1\fi\ifluatex 1\fi=0 % if pdftex
  \usepackage[T1]{fontenc}
  \usepackage[utf8]{inputenc}
  \usepackage{textcomp} % provide euro and other symbols
\else % if luatex or xetex
  \usepackage{unicode-math}
  \defaultfontfeatures{Scale=MatchLowercase}
  \defaultfontfeatures[\rmfamily]{Ligatures=TeX,Scale=1}
\fi
% Use upquote if available, for straight quotes in verbatim environments
\IfFileExists{upquote.sty}{\usepackage{upquote}}{}
\IfFileExists{microtype.sty}{% use microtype if available
  \usepackage[]{microtype}
  \UseMicrotypeSet[protrusion]{basicmath} % disable protrusion for tt fonts
}{}
\makeatletter
\@ifundefined{KOMAClassName}{% if non-KOMA class
  \IfFileExists{parskip.sty}{%
    \usepackage{parskip}
  }{% else
    \setlength{\parindent}{0pt}
    \setlength{\parskip}{6pt plus 2pt minus 1pt}}
}{% if KOMA class
  \KOMAoptions{parskip=half}}
\makeatother
\usepackage{xcolor}
\IfFileExists{xurl.sty}{\usepackage{xurl}}{} % add URL line breaks if available
\IfFileExists{bookmark.sty}{\usepackage{bookmark}}{\usepackage{hyperref}}
\hypersetup{
  pdftitle={Análisis de Diversidad},
  pdfauthor={Camilo Vélez},
  hidelinks,
  pdfcreator={LaTeX via pandoc}}
\urlstyle{same} % disable monospaced font for URLs
\usepackage[margin=1in]{geometry}
\usepackage{color}
\usepackage{fancyvrb}
\newcommand{\VerbBar}{|}
\newcommand{\VERB}{\Verb[commandchars=\\\{\}]}
\DefineVerbatimEnvironment{Highlighting}{Verbatim}{commandchars=\\\{\}}
% Add ',fontsize=\small' for more characters per line
\usepackage{framed}
\definecolor{shadecolor}{RGB}{248,248,248}
\newenvironment{Shaded}{\begin{snugshade}}{\end{snugshade}}
\newcommand{\AlertTok}[1]{\textcolor[rgb]{0.94,0.16,0.16}{#1}}
\newcommand{\AnnotationTok}[1]{\textcolor[rgb]{0.56,0.35,0.01}{\textbf{\textit{#1}}}}
\newcommand{\AttributeTok}[1]{\textcolor[rgb]{0.77,0.63,0.00}{#1}}
\newcommand{\BaseNTok}[1]{\textcolor[rgb]{0.00,0.00,0.81}{#1}}
\newcommand{\BuiltInTok}[1]{#1}
\newcommand{\CharTok}[1]{\textcolor[rgb]{0.31,0.60,0.02}{#1}}
\newcommand{\CommentTok}[1]{\textcolor[rgb]{0.56,0.35,0.01}{\textit{#1}}}
\newcommand{\CommentVarTok}[1]{\textcolor[rgb]{0.56,0.35,0.01}{\textbf{\textit{#1}}}}
\newcommand{\ConstantTok}[1]{\textcolor[rgb]{0.00,0.00,0.00}{#1}}
\newcommand{\ControlFlowTok}[1]{\textcolor[rgb]{0.13,0.29,0.53}{\textbf{#1}}}
\newcommand{\DataTypeTok}[1]{\textcolor[rgb]{0.13,0.29,0.53}{#1}}
\newcommand{\DecValTok}[1]{\textcolor[rgb]{0.00,0.00,0.81}{#1}}
\newcommand{\DocumentationTok}[1]{\textcolor[rgb]{0.56,0.35,0.01}{\textbf{\textit{#1}}}}
\newcommand{\ErrorTok}[1]{\textcolor[rgb]{0.64,0.00,0.00}{\textbf{#1}}}
\newcommand{\ExtensionTok}[1]{#1}
\newcommand{\FloatTok}[1]{\textcolor[rgb]{0.00,0.00,0.81}{#1}}
\newcommand{\FunctionTok}[1]{\textcolor[rgb]{0.00,0.00,0.00}{#1}}
\newcommand{\ImportTok}[1]{#1}
\newcommand{\InformationTok}[1]{\textcolor[rgb]{0.56,0.35,0.01}{\textbf{\textit{#1}}}}
\newcommand{\KeywordTok}[1]{\textcolor[rgb]{0.13,0.29,0.53}{\textbf{#1}}}
\newcommand{\NormalTok}[1]{#1}
\newcommand{\OperatorTok}[1]{\textcolor[rgb]{0.81,0.36,0.00}{\textbf{#1}}}
\newcommand{\OtherTok}[1]{\textcolor[rgb]{0.56,0.35,0.01}{#1}}
\newcommand{\PreprocessorTok}[1]{\textcolor[rgb]{0.56,0.35,0.01}{\textit{#1}}}
\newcommand{\RegionMarkerTok}[1]{#1}
\newcommand{\SpecialCharTok}[1]{\textcolor[rgb]{0.00,0.00,0.00}{#1}}
\newcommand{\SpecialStringTok}[1]{\textcolor[rgb]{0.31,0.60,0.02}{#1}}
\newcommand{\StringTok}[1]{\textcolor[rgb]{0.31,0.60,0.02}{#1}}
\newcommand{\VariableTok}[1]{\textcolor[rgb]{0.00,0.00,0.00}{#1}}
\newcommand{\VerbatimStringTok}[1]{\textcolor[rgb]{0.31,0.60,0.02}{#1}}
\newcommand{\WarningTok}[1]{\textcolor[rgb]{0.56,0.35,0.01}{\textbf{\textit{#1}}}}
\usepackage{graphicx,grffile}
\makeatletter
\def\maxwidth{\ifdim\Gin@nat@width>\linewidth\linewidth\else\Gin@nat@width\fi}
\def\maxheight{\ifdim\Gin@nat@height>\textheight\textheight\else\Gin@nat@height\fi}
\makeatother
% Scale images if necessary, so that they will not overflow the page
% margins by default, and it is still possible to overwrite the defaults
% using explicit options in \includegraphics[width, height, ...]{}
\setkeys{Gin}{width=\maxwidth,height=\maxheight,keepaspectratio}
% Set default figure placement to htbp
\makeatletter
\def\fps@figure{htbp}
\makeatother
\setlength{\emergencystretch}{3em} % prevent overfull lines
\providecommand{\tightlist}{%
  \setlength{\itemsep}{0pt}\setlength{\parskip}{0pt}}
\setcounter{secnumdepth}{-\maxdimen} % remove section numbering
\usepackage{booktabs}
\usepackage{longtable}
\usepackage{array}
\usepackage{multirow}
\usepackage{wrapfig}
\usepackage{float}
\usepackage{colortbl}
\usepackage{pdflscape}
\usepackage{tabu}
\usepackage{threeparttable}
\usepackage{threeparttablex}
\usepackage[normalem]{ulem}
\usepackage{makecell}
\usepackage{xcolor}

\title{Análisis de Diversidad}
\author{Camilo Vélez}
\date{14/8/2021}

\begin{document}
\maketitle

{
\setcounter{tocdepth}{2}
\tableofcontents
}
\hypertarget{marco-general}{%
\subsection{Marco General}\label{marco-general}}

~~~~En general, la diversidad es una medida para cuantificar el número
de diferentes estados en un sistema. En el caso particular de las
comunidades biológicas, estos estados suelen ser especies pero también
pueden referirse a géneros, familias, OTU's (Operational Taxonomic Unit)
o grupos funcionales. En ecología, la diversidad es considerada una
``propiedad emergente'' de las comunidades y por lo tanto actúa a nivel
de las comunidades y no de las especies.

~~~~ La diversidad tiene dos componentes: \textbf{riqueza de especies}
(número de especies en una comunidad), y \textbf{la uniformidad o
equilibrio de especies} (forma en la que se distribuye la abundancia de
especies más comunes y las consideradas ``raras''). En una comunidad
pefectamente \emph{uniforme}, cada especie está representada por el
mismo número de individuos y por tanto hay una alta probabilidad de
encontrar una especie nueva durante un muestreo aleatorio. Si es
altamente desigual, la comunidad es dominada por una o pocas especies
mientras que las otras son consideradas como \emph{raras}. En este caso,
la probabilidad de encontrar una especie nueva se reduce drásticamente.
Una comunidad será más diversa si, además de mostrar un alto número de
especies, presenta una alta uniformidad.

~~~~Es interesante destacar que una comunidad biológica uniforme y otra
desigual son funcionalmente distintas. En la primera, las interacciones
ocurren principalmente entre individuos de distintas especies, por lo
que en ella prevalecen las interacciones interespecíficas. En una
comunidad desigual predominan las interacciones intraespecíficas ya que
los individuos que interactúan en ella pertenecen a la misma especie.

~~~~Hay varios índices de diversidad que difieren en cuanto a si estos
ponen más énfasis en la riqueza de especies o en la uniformidad de la
comunidad. La riqueza de especies no tiene en cuenta la equidad; para el
índice de Simpson es fundamental. Si bien el cálculo de los índices de
diversidad es una tarea relativamente sencilla, es necesario conocer
como operan y cuales son sus limitaciones a fin de interpretar
adecuadamente el significado de cada uno de ellos.

~~~~La riqueza de especies y la equitabilidad de una comunidad son
valores teóricos, ya que ellos se estiman a partir de muestreos, y por
definición los muestreos siempre son incompletos. Los estimadores de
diversidad dependen del esfuerzo de muestreo, entre mayor sea este más
ajustados serán los índices.

~~~~ Whittaker introduce los términos de alfa, beta y gamma diversidad a
fin de estimar la variedad biológica de un ecosistema en distintas
escalas greográficas. La alfa-diversidad mide la riqueza de especies a
nivel local, mientras que la beta-diversidad estima la tasa de cambio de
especies de dos comunidades biológicas adjacentes dispuestas a lo largo
de un gradiente espacial, temporal o ambiental (refleja el grado de
heterogeneidad de un conjunto de comunidades). La alfa y beta diversidad
son estimaciones independientes: una comunidad puede mostrar una
alfa-diversidad media alta y a su vez presentar una beta-diversidad
media baja, y viceversa.

\hypertarget{carga-de-las-bases-de-datos}{%
\subsubsection{Carga de las bases de
datos}\label{carga-de-las-bases-de-datos}}

Conteo de diatomeas: ríos Colorado (RC), Negro (RN) y Chubut (CH)

\begin{Shaded}
\begin{Highlighting}[]
\NormalTok{especiesRC <-}\StringTok{ }\KeywordTok{read.table}\NormalTok{(}\StringTok{"clipboard"}\NormalTok{, }\DataTypeTok{header =}\NormalTok{ T, }\DataTypeTok{row.names =} \DecValTok{1}\NormalTok{, }\DataTypeTok{sep =}\StringTok{"}\CharTok{\textbackslash{}t}\StringTok{"}\NormalTok{)}
\NormalTok{especiesRN <-}\StringTok{ }\KeywordTok{read.table}\NormalTok{(}\StringTok{"clipboard"}\NormalTok{, }\DataTypeTok{header =}\NormalTok{ T, }\DataTypeTok{row.names =} \DecValTok{1}\NormalTok{, }\DataTypeTok{sep =}\StringTok{"}\CharTok{\textbackslash{}t}\StringTok{"}\NormalTok{)}
\NormalTok{especiesCH <-}\StringTok{ }\KeywordTok{read.table}\NormalTok{(}\StringTok{"clipboard"}\NormalTok{, }\DataTypeTok{header =}\NormalTok{ T, }\DataTypeTok{row.names =} \DecValTok{1}\NormalTok{, }\DataTypeTok{sep =}\StringTok{"}\CharTok{\textbackslash{}t}\StringTok{"}\NormalTok{)}
\end{Highlighting}
\end{Shaded}

\hypertarget{riqueza-especuxedfica-s}{%
\subsubsection{Riqueza específica (S)}\label{riqueza-especuxedfica-s}}

~~~~Es una medida que está relacionada con el número de especies
presentes en una comunidad.

\begin{Shaded}
\begin{Highlighting}[]
\NormalTok{S.RC <-}\StringTok{ }\KeywordTok{specnumber}\NormalTok{(especiesRC)}
\KeywordTok{barplot}\NormalTok{(S.RC)}
\end{Highlighting}
\end{Shaded}

\includegraphics{Análisis-de-Diversidad_files/figure-latex/unnamed-chunk-2-1.pdf}

\begin{Shaded}
\begin{Highlighting}[]
\NormalTok{S.RN <-}\StringTok{ }\KeywordTok{specnumber}\NormalTok{(especiesRN)}
\KeywordTok{barplot}\NormalTok{(S.RN)}
\end{Highlighting}
\end{Shaded}

\includegraphics{Análisis-de-Diversidad_files/figure-latex/unnamed-chunk-3-1.pdf}

\begin{Shaded}
\begin{Highlighting}[]
\NormalTok{S.CH <-}\StringTok{ }\KeywordTok{specnumber}\NormalTok{(especiesCH)}
\KeywordTok{barplot}\NormalTok{(S.CH)}
\end{Highlighting}
\end{Shaded}

\includegraphics{Análisis-de-Diversidad_files/figure-latex/unnamed-chunk-4-1.pdf}

\end{document}
